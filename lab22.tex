\documentclass[a4paper,12pt]{article} 

\usepackage{cmap}                   
\usepackage{mathtext}               
\usepackage[T2A]{fontenc}           
\usepackage[utf8]{inputenc}         
\usepackage[english,russian]{babel} 

\usepackage{amsmath,amsfonts,amssymb,amsthm,mathtools} % математика

\usepackage{euscript} 
\usepackage{mathrsfs} 

\setcounter{page}{603}

\begin{document} 

\[
    I_n=\int\limits_0^{\pi/2} \sin^nx \, dx = 
    \int\limits_0^{\pi/2} \cos^nx \, dx =\begin{cases}
        \frac{(n-1)!!}{n!!}\frac{\pi}{2} &\text{при n чётном,} \\
        \frac{(n-1)!!}{n!!} &\text{при n нечётном.}
    \end{cases}
    \tag{26.7}
\]

Из формулы (26.7) легко получается \textit{формула Валлиса}\footnote{Дж. Валлис (1616 --- 1703) --- английский математик.}, которая понадобится в дальнейшем:
\[ \frac{\pi}{2} = \lim_{n\to \infty} \frac{1}{2n + 1} \left[ \frac{(2n)!!}{(2n-1)!!} \right]^2. \tag{26.8}\]
Докажем её. Интегрируя неравенство
\[ \sin^{2n + 1}x \leqslant \sin^{2n}x \leqslant \sin^{2n - 1}x, \, 0 \leqslant x \leqslant \frac{\pi}{2},\]
по отрезку $[0, \pi/2]$
\[ \int\limits_0^{\pi/2} \sin^{2n + 1}x \, dx \leqslant \int\limits_0^{\pi/2} \sin^{2n}x \, dx \leqslant \int\limits_0^{\pi/2} \sin^{2n - 1}x \, dx, \, 0 \leqslant x \leqslant \frac{\pi}{2},\]
Согласно (26.7),
\[ \frac{(2n)!!}{(2n+1)!!} \leqslant \frac{(2n-1)!!}{(2n)!!}\frac{\pi}{2} 
\leqslant \frac{(2n-2)!!}{(2n-1)!!},
\]
откуда
\[ x_n \overset{def}{=} \frac{1}{2n + 1} \left[ \frac{(2n)!!}{(2n-1)!!} \right]^2 \leqslant \frac{\pi}{2} \leqslant \frac{1}{2n} \left[ \frac{(2n)!!}{(2n-1)!!} \right]^2 \overset{def}{=} y_n.   \tag{26.9}\]
В силу этого неравенства, при $ n \to \infty$
\[ x_n - y_n = \frac{1}{2n} \frac{1}{2n + 1} \left[ \frac{(2n)!!}{(2n-1)!!} \right]^2 \leqslant \frac{1}{2n}\frac{\pi}{2} \to 0, \]
поэтому $ \lim\limits_{n \to \infty} (y_n - x_n) = 0$, 
т.е. длины отрезков $[x_n,y_n]$, содержащих $ \frac{\pi}{2}$, стремятся к нулю и, следовательно, 
$ \lim\limits_{n \to \infty} x_n = \frac{\pi}{2}$, $ \lim\limits_{n \to \infty} y_n = \frac{\pi}{2}$. Первое из этих равенств, 
согласно определению $ x_n $ (см. (26.9)), и означает справедливость формулы Валлиса.



\section*{26.3* Вторая теорема о среднем значении для определённого интеграла}

\textbf{Лемма 1.} \textit{Пусть f --- непрерывная, а g --- возрастающая неотрицательная непрерывно дифференцируемая на отрезке $ [a, b] $ функция. 
Тогда существует такая точка $ \xi \in  [a, b] $, что} 
\[ \int\limits_a^b g(x)f(x)\,dx = g(b) \int\limits_{\xi}^b f(x)\,dx \tag{26.10}\]

\textit{Доказательство.} Рассмотрим функцию
\[ F(x) \overset{def}{=} g(b) \int\limits_x^b f(t)\,dt, \, a \leqslant x \leqslant b \tag{26.11}\]

Функция F, являясь интегралом с переменным нижним пределом интегрирования от интегрируемой (даже непрерывной) функции f, непрерывна на отрезке $ [a, b] $ 
и поэтому достигает на нём своего наибольшего и наименьшего значения. Если

\[ m = \min_{[a, b]} F(x), \, M = \max_{[a, b]} F(x), \tag{26.12}\]
то, очевидно,
\[ m \leqslant F(x) \leqslant M, \, x \in  [a, b].  \tag{26.13}\]

\[ \int\limits_a^b g(x)f(x) \, dx = -\int\limits_a^b g(x) \, dF(x) = -g(x)F(x)\bigg|_a^b + \int\limits_a^b F(x)g'(x) \, dx  = \] 
\[ =  g(a)F(a) + \int\limits_a^b F(x)g'(x) \, dx, \tag{26.14}\] 
так как, в силу (26.11), $ F(b) = 0 $.

Функция $ g $ возрастающая, поэтому имеем $ g'(x) \geqslant 0 $ для всех $ x \in [a, b] $. Применив это неравенство, неравенство (26.13) 
и заметив, что из неотрицательности $ g $ на $[a,b]$ следует, в частности, что $ g(a) \geqslant 0 $, получим оценки

\[ g(a)F(a) + \int\limits_a^b F(x)g'(x) \, dx \leqslant Mg(a) + M\int\limits_a^b g'(x) \, dx = \] 
\[ = Mg(a) + M[g(b) - g(a)] = Mg(b), \] 
\[ g(a)F(a) + \int\limits_a^b F(x)g'(x) \, dx \geqslant mg(a) + m[g(b) - g(a)] = mg(b). \] 
Таким образом (см. (26.14)), имеем
\[ mg(b) \leqslant \int\limits_a^b g(x)f(x) \, dx \leqslant Mg(b). \] 
Если $ g(b) = 0 $, то из неотрицательности и возрастания функции g следует, что 
$ g(x) \equiv 0 $ на $ [a,b] $. В этом случае формула (26.10) справедлива при любом выборе $ \xi \in  [a, b] $.

Если же $ g(x) > 0 $, то

\[ m \leqslant \frac{1}{g(b)}\int\limits_a^b g(x)f(x) \, dx \leqslant M. \] 

Непрерывная на отрезке $ [a,b] $ функция $ F $ принимает на этом отрезке любое значение, лежащее между её минимальным значением m и 
максимальным $ M $  (см. (26.12)), поэтому существует такая точка $ \xi \in  [a, b] $, что

\[ F(\xi) = \frac{1}{g(b)}\int\limits_a^b g(x)f(x) \, dx. \] 
В силу определения (26.11), это и есть формула (26.10). $ \square $

\end{document}



